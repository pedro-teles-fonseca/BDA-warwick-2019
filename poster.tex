
\PassOptionsToPackage{table, usenames, dvipsnames}{xcolor}

\documentclass[20pt, margin=1in,innermargin=-4.5in,blockverticalspace=-0.25in]{tikzposter}

\geometry{paperwidth=42in,paperheight=30in, bottom=0cm}

\usepackage[utf8]{inputenc}
\usepackage{amsmath}
\usepackage{amsfonts}
\usepackage{amsthm}
\usepackage{amssymb}
\usepackage{bm}
\usepackage{mathrsfs}
\usepackage{graphicx}
\usepackage{adjustbox}
\usepackage{theme}
\usepackage{array, multirow}
\usepackage{multicol}
\usepackage[inline]{enumitem}
\usepackage{changepage}
\usepackage[hidelinks]{hyperref}	

\newlist{inlinelist}{enumerate*}{1}
\setlist[inlinelist]{label=(\roman*)}

% set theme parameters
\tikzposterlatexaffectionproofoff
\usetheme{EmoryTheme}
\usecolorstyle{EmoryStyle}
\colorlet{blocktitlebgcolor}{Maroon}

% bibliography options

\usepackage[style=authoryear,
 bibencoding=utf8, minnames=1, maxnames=3,
maxbibnames=4,backref=false, natbib=true, dashed=false, terseinits=true,
%firstinits=true,
uniquename=false, uniquelist=true, labeldateparts=true, 
doi=false, isbn=false, eprint = false, url = false, % desativei os url e eprints - confirmar se é para manter!
natbib=true, backend=biber]{biblatex}

\DefineBibliographyStrings{english}{
    backrefpage = {Cited on page},
    backrefpages = {Cited on pages},
}

\AtBeginBibliography{\footnotesize}

% Change the default bibliography formatting to be more "statistical"
\DeclareFieldFormat{url}{\url{#1}}
\DeclareFieldFormat[article]{pages}{#1}
\DeclareFieldFormat[inproceedings]{pages}{\lowercase{pp.}#1}
\DeclareFieldFormat[incollection]{pages}{\lowercase{pp.}#1}
\DeclareFieldFormat[article]{volume}{\mkbibbold{#1}}
\DeclareFieldFormat[article]{number}{\mkbibparens{#1}}
\DeclareFieldFormat[article]{title}{\MakeCapital{#1}}
\DeclareFieldFormat[article]{url}{}
\DeclareFieldFormat[book]{url}{}
\DeclareFieldFormat[inbook]{url}{}
\DeclareFieldFormat[thesis]{url}{}
\DeclareFieldFormat[thesis]{url}{}
\DeclareFieldFormat[incollection]{url}{}
\DeclareFieldFormat[inproceedings]{url}{}
\DeclareFieldFormat[inproceedings]{title}{#1}
\DeclareFieldFormat{shorthandwidth}{#1}
\DeclareFieldFormat[thesis]{citetitle}{#1}
\DeclareFieldFormat[thesis]{title}{#1} 

% No dot before number of articles
\usepackage{xpatch}
\xpatchbibmacro{volume+number+eid}{\setunit*{\adddot}}{}{}{}

% Remove In: for an article.
\renewbibmacro{in:}{%
  \ifentrytype{article}{}{%
  \printtext{\bibstring{in}\intitlepunct}}}
  
% Get rid of notes in citations
\AtEveryBibitem{\clearfield{note}}

% Get rid of months in citations
\AtEveryBibitem{\clearfield{month}}
\AtEveryCitekey{\clearfield{month}}
\raggedbottom

\addbibresource{references.bib}

% Title section

\title{Bayesian Digit Analysis}
\author{Pedro Fonseca}
\institute{ISEG Lisbon School of Economics and Management, University of Lisbon\\
            CEMAPRE -- Center for Applied Mathematics and Economics}

% begin document
\begin{document}
\maketitle

\node[anchor=west] at (TP@title.west) {\includegraphics[width=0.15\textwidth]{images/logos2.jpg}};
\node[anchor=east] at (TP@title.east) {\includegraphics[width=0.25\textwidth]{images/cemapre.png}};

\centering

\begin{columns}
 
 \column{0.32}
    
\block{Background and Motivation}{

Digit analysis consists in using empirical regularities regarding the occurrence of digits in numbers to screen numerical datasets for erroneous or fraudulent data. Most of the digit analysis literature relies on Benford's law, a logarithmically-decaying pattern in leading digits frequencies that was first discovered by \citet{newcomb1881note} and later supported with empirical evidence by \citet{benford1938}. The first rigorous proof of the emergence of Benford's law is due to \citet{hill1995derivation}. According to Benford's law:

\vspace{.5cm}

\begin{itemize}
\item[] $P(\text{First Digit} = d_1) = \operatorname{log \left( 1 + \frac{1}{d_1}\right)},  \, d_1 \in \{1, \dots, 9 \} \,\,\,\,\,\,\,\,\,\,\,\,\,\,\,\,\,\,\,\,\,\,\,\,\,\,\,\,\,\,\,\,\,\,\,\,\,\,\,\,\,\,\,\,\,\,\,\,\,$ (BL1)  
\item[] $P(\text{Second Digit} = d_2) =  \sum_{d_1=1}^{9} \operatorname{log \left( 1 + \frac{1}{10 \, d_1 + d_2}\right)}, \,  d_2 \in \{0, \dots, 9 \} \, \, \, \, \, \,  \, \, \, \, \, \, $ (BL2)
\end{itemize}

\vspace{.5cm}

Digit analysis usually requires testing point null hypotheses. In that context, classical significance tests are known to over-reject the null in large samples due to the high levels of power they attain, as the acceptance region shrinks with sample size for a given significance level \citep{ley1996peculiar}. Consequently, a large proportion of legit datasets is likely to be classified as fraudulent. Any deviation from the null, no matter how tiny, can produce small $p$-values and be declared statistically significant if the sample is large enough \citep{wasserstein2016asa}. Hence, the usefulness and interpretation of classical $p$-values are drastically affected by sample size \citep{pericchiTorres2011}.
}
        
\block{Methodology Part I}{

\vspace{-1em}

\begin{itemize}

\item Bayesian model selection: $\text{Multinomial} \land \text{Dirichlet}$ model

\begin{itemize}
\item Likelihood: $\bm{x}|\bm{\theta} \sim \mathcal{M}_k(N,\bm{\theta})$ 
\item Hypotheses:  $ H_0: \bm{\theta}=\bm{\theta_0} \,\, \text{vs} \,\,  H_1: \bm{\theta} \neq \bm{\theta_0}$. 
\item Prior: $\pi(\bm{\theta}|H_0)={1_{\bm{\theta_0}}(\bm{\theta_0})}$ and $\bm{\theta}|H_1 \sim \operatorname{Dir}_k(\bm{\alpha})$ 

\begin{itemize}
\item[(i)] $\bm{\alpha} = \bm{1}$: Uniform prior
\item[(ii)] $\bm{\alpha} = c \, \bm{\theta_0}$, $c>0$: Dirichlet prior centred on $\bm{\theta_0}$
\end{itemize}

\item Marginal density: $m_i(\bm{x})=\int_{\Theta_i} f(\bm{x}|\bm{\theta})\pi(\bm{\theta}|H_i) \, d\bm{\theta}, \, i=0,1$

\begin{inlinelist}
\item $m_0(\bm{x})=\frac{N!}{\prod_{i=1}^{k+1}x_i!}\prod_{i=1}^{k+1}{\theta_{0i}}^{x_i} \, \, $ 
\item $m_1(\bm{x})= \frac{N!}{\prod_{i=1}^{k+1}x_i!}\frac{B(\bm{\alpha}+\bm{x})}{B(\bm{\alpha})}$ 
\end{inlinelist}
\item Bayes factor: $B_{01}(\bm{x}) = \frac{m_0(\bm{x})}{m_1(\bm{x})} = \frac{\prod_{i=1}^{k+1}{(\theta_{0i}}^{x_i}) \prod_{i=1}^{k+1}[\Gamma(\alpha_i)] \Gamma [\sum_{i=1}^{k+1}(\alpha_i+x_i)]}{\Gamma(\sum_{i=1}^{k+1} \alpha_i) \prod_{i=1}^{k+1}\Gamma(\alpha_i+x_i)}$
\end{itemize}	

\item Bayesian model selection: $\text{Binomial} \land \text{Beta}$ model

\begin{itemize}
\item Likelihood: $x \sim \text{Bin}(N, \theta)$
\item Hypotheses: $ H_0: {\theta}={\theta_0} \,\,\,\, \text{vs} \,\,\,\,  H_1: {\theta} \neq \theta_0$
\item Prior: $\pi(\theta|H_0)={1_{\theta_0}(\theta)}$ and $\theta| H_1 \sim \operatorname{Beta}(a, b) $

\begin{itemize}
\item[(i)]  $(a, b) = (1, 1)$: Uniform prior
\item[(ii)] $(a, b) = (c \, \theta_0, c - c \, \theta_0)$, $c>0$: Beta prior centred on $\theta_0$
\end{itemize}

\item Marginal density: $m_i(x)= \int_0^1 f(x|\theta)\pi(\theta|H_i)d \theta, \,  i=0, 1$

\begin{inlinelist}
\item $m_0(x)=\binom{N}{x} {\theta_0}^{x}{(1-\theta_0)}^{N-x} \, \, $ 
\item $m_1(x) = \binom{N}{x}\frac{B(x+a, n-x+b)}{B(a,b)}$
\end{inlinelist}
\item Bayes factor: $B_{01}(x) = \frac{m_0(x)}{m_1(x)}  = \frac{\theta_0(1-\theta_0)^{N-x}\,\Gamma(a)\,\Gamma(b)\,\Gamma(n+a+b)}{\Gamma(a+b)\,\Gamma(n+a-x)\,\Gamma(x+a)}$ 

\end{itemize}

\item $\bm{\theta_0}$ represents agreement to either BL1 or BL2 digit probabilities. $\theta_0$ represents agreement to one of BL1 or BL2 digit probabilities when considered individually. 

\end{itemize}

}

\column{0.36}

\block{Methodology Part II}{

\vspace{-1em}

\begin{itemize}
\item $\bm{x}$ represents the counts of first (or second) digits, and $x$ the count of one particular digit.
\item The choice of $c$ defines how informative the priors are. Considering only $c\in\mathbb{N}$:
\begin{itemize}
\item $c=1$: the least informative priors centred on the null parameter values
\item $c=22$ for BL1 and $c=12$ for BL2: the least informative uni-modal priors centred on the null parameter values
\end{itemize}

\item Prior model probabilities: $P(H_0) = \pi_0$ and $P(H_1) = 1 - \pi_0$
\item Posterior model probabilities: $P(H_0|\text{data})={\left(1+\frac{1-\pi_0}{\pi_0}\,B_{01}{(\text{data})}^{-1}\right)}^{-1}$
\begin{itemize}
\item \citet{bergerSelke1987} consider $\pi_0 = \frac{1}{2}$ to be the objective choice.
\end{itemize}

\item P-value Calibration \citep{sellke2001}: 

\begin{itemize}

\item Let $p_{obs}$ be a $p$-value on a classical test statistic. The lower bound on the Bayes factor in favour of $H_0$ is: $\underline{B}_{01}(p_{obs}) = -e \, p_{obs} \, \operatorname{ln}(p_{obs})$ if $p_{obs}<\frac{1}{e}$ and $\underline{B}_{01}(p_{obs})={1}$ otherwise.

\item With $\pi_0 = \frac{1}{2}$, the lower bound on $P(H_0|\text{data})$ is: $\underline{P}(H_0|{{p_{obs}}}) =  {\left(1+{\underline{B}_{01}(p_{obs})}^{-1}\right)}^{-1}$

\item We will use this calibration on $p$-values from $\chi^2$-tests and Nigrini's (\cite*{nigrini2012benford}) $z$-tests.
\end{itemize}

\end{itemize}

}    
    
\block{Data}{

The data was downloaded from the {\href{http://ec.europa.eu/eurostat/data/database}{Eurostat website}} (Data/Database by themes/Economy and finance/Government statistics/Government finance statistics/Government deficit and debt/Government deficit/surplus, debt and associated data). Each countries' sample consists in the aggregation of all the numbers from the 38 tables in this category. This category of data is related to public deficit and public debt, which are variables that are important to the Stability and Growth Pact criteria. \citet*{rauch2011fact} warn that macroeconomic statistics can be used by governments to portray a more favourable picture of their countries economic situation. The pressure for European Union's governments to comply with the Stability and Growth Pact criteria is an additional incentive for them to manipulate macroeconomic statistics. We included data from 1999 to 2015, and only countries that joined the Eurozone prior to 2006 were considered so that at least 10 years of data were available. Our samples meet the conditions that according to \citet{hill1995derivation} typically result in the emergence of Benford's law. 
 
}
    
\block{Results: $\text{Multinomial} \land \text{Dirichlet}$ Model \& $\chi^2$-Test}{

\vspace{-1em}

\begin{center}

\resizebox{\linewidth}{!}{

\begin{tabular}{|l|c|c|c|c|c|c|c|c|c|c|c|c|c|c|c|}

\cline{2-13}

\multicolumn{1}{l|}{\textbf{}} & \multicolumn{6}{|c|}{\textbf{BL1}} & \multicolumn{6}{|c|}{\textbf{BL2}}\\

\hline

\multicolumn{1}{|c|}{{\multirow{2}{*}{Country}}} &  \multicolumn{1}{c|}{{\multirow{2}{*}{$N$}}} & \multicolumn{3}{|c|}{$P(H_0| \bm{x})$} &  \multicolumn{1}{c|}{{\multirow{2}{*}{${\underline{P}}{(H_0|\bm{x})}$}}} & \multicolumn{1}{c|}{{\multirow{2}{*}{${p_{\text{{obs}}}}$}}} &  \multicolumn{1}{c|}{{\multirow{2}{*}{$N$}}} &\multicolumn{3}{|c|}{$P(H_0| \bm{x})$}  &  \multicolumn{1}{c|}{{\multirow{2}{*}{${\underline{P}}{(H_0|\bm{x})}$}}} & \multicolumn{1}{c|}{{\multirow{2}{*}{${p_{\text{{obs}}}}$}}}\\
 
 \cline{3-5}  \cline{9-11} 
 
 &  & \multicolumn{1}{c|}{$\bm{\alpha} =\bm{1}$}  &  \multicolumn{1}{c|}{$\bm{\alpha} =\bm{\theta_0}$}  &  \multicolumn{1}{c|}{$\bm{\alpha}=22 \, \bm{\theta_0}$}   & & & & \multicolumn{1}{c|}{$\bm{\alpha} =\bm{1}$}  &  \multicolumn{1}{c|}{$\bm{\alpha} =\bm{\theta_0}$}  &  \multicolumn{1}{c|}{$\bm{\alpha}=12 \, \bm{\theta_0}$}  &  & \\

\hline
Austria&$ 619$&$0.001$&$0.988$&$0.000$&$0.000$&$0.000$&$ 614$&$1.000$&$1.000$&$1.000$&$0.357$&$0.082$\\
Belgium&$ 604$&$0.016$&$1.000$&$0.000$&$0.000$&$0.000$&$ 604$&$1.000$&$1.000$&$1.000$&$0.481$&$0.236$\\
Finland&$ 605$&$0.919$&$1.000$&$0.200$&$0.000$&$0.000$&$ 605$&$1.000$&$1.000$&$1.000$&$0.331$&$0.068$\\
France&$ 600$&$0.039$&$1.000$&$0.001$&$0.000$&$0.000$&$ 600$&$1.000$&$1.000$&$0.999$&$0.069$&$0.005$\\
Germany&$ 612$&$0.942$&$1.000$&$0.113$&$0.002$&$0.000$&$ 610$&$1.000$&$1.000$&$1.000$&$0.453$&$0.175$\\
Greece&$ 629$&$0.885$&$1.000$&$0.067$&$0.002$&$0.000$&$ 627$&$1.000$&$1.000$&$1.000$&$0.152$&$0.016$\\
Ireland&$ 616$&$0.004$&$0.997$&$0.000$&$0.000$&$0.000$&$ 616$&$1.000$&$1.000$&$1.000$&$0.500$&$0.387$\\
Italy&$ 625$&$0.945$&$1.000$&$0.110$&$0.002$&$0.000$&$ 608$&$1.000$&$1.000$&$1.000$&$0.346$&$0.075$\\
Luxembourg&$ 602$&$0.000$&$0.000$&$0.000$&$0.000$&$0.000$&$ 602$&$1.000$&$1.000$&$0.999$&$0.089$&$0.007$\\
Netherlands&$ 596$&$1.000$&$1.000$&$0.986$&$0.105$&$0.009$&$ 596$&$1.000$&$1.000$&$1.000$&$0.313$&$0.060$\\
Portugal&$ 617$&$0.000$&$0.000$&$0.000$&$0.000$&$0.000$&$ 617$&$1.000$&$1.000$&$1.000$&$0.461$&$0.189$\\
Spain&$ 535$&$0.808$&$1.000$&$0.018$&$0.002$&$0.000$&$ 530$&$1.000$&$1.000$&$1.000$&$0.500$&$0.855$\\
\hline
Pooled Sample&$7260$&$0.999$&$1.000$&$0.912$&$0.000$&$0.000$&$7229$&$1.000$&$1.000$&$1.000$&$0.317$&$0.061$\\
\hline
\end{tabular}

}

\end{center}

 }

\column{0.32}

\block{Results: $\text{Binomial} \land \text{Beta}$ Model \& Z-Test}{

\vspace{-1em}

\begin{center}

\resizebox{\linewidth}{!}{

\begin{tabular}{|c|c|c|c|c|c|c|c|c|c|c|c|c|c|c|}

\cline{2-7} \cline{10-15} 

\multicolumn{1}{c}{} &\multicolumn{1}{|c|}{\multirow{3}{*}{Digit}} & \multicolumn{3}{c|}{$P{(H_0|{x})}$} &  \multicolumn{1}{c|}{\multirow{3}{*}{${\underline{P}}{(H_0|{x})}$}} & \multicolumn{1}{c|}{\multirow{3}{*}{${p_{\text{\textbf{obs}}}}$}} & \multicolumn{1}{c}{}&\multicolumn{1}{c|}{}& \multicolumn{1}{c|}{\multirow{3}{*}{Digit}} & \multicolumn{3}{c|}{$P{(H_0|{x})}$} &  \multicolumn{1}{c|}{\multirow{3}{*}{${\underline{P}}{(H_0|{x})}$}} & \multicolumn{1}{c|}{\multirow{3}{*}{${p_{\text{\textbf{obs}}}}$}}\\

\cline{3-5} \cline{11-13} 

\multicolumn{1}{c|}{}&&\multicolumn{1}{c|}{$a=1$}&\multicolumn{1}{c|}{$a = \theta_0$}&\multicolumn{1}{c|}{$a = 22 \, \theta_0$}& &&\multicolumn{1}{c}{}&  & &\multicolumn{1}{c|}{$a=1$}&\multicolumn{1}{c|}{$a = \theta_0$}&\multicolumn{1}{c|}{$a = 22 \, \theta_0$}& & \\

\multicolumn{1}{c|}{} & &\multicolumn{1}{c|}{$b=1$}&\multicolumn{1}{c|}{$b = 1-\theta_0$}&\multicolumn{1}{c|}{$b = 22-22 \, \theta_0$}& &&\multicolumn{1}{c}{}&  & &\multicolumn{1}{c|}{$b=1$}&\multicolumn{1}{c|}{$b = 1-\theta_0$}&\multicolumn{1}{c|}{$b = 22-22 \, \theta_0$}& &\\

\cline{1-7} \cline{9-15} 
\multirow{9}{*}{\rotatebox[origin=c]{90}{\textbf{Austria BL1}}}&1&$0.542$&$0.949$&$1.00$&$0.170$&$0.019$&&\multirow{9}{*}{\rotatebox[origin=c]{90}{\textbf{Belgium BL1}}}&1&$0.003$&$0.036$&$1.00$&$0.000$&$0.000$\\
&2&$0.951$&$0.999$&$1.00$&$0.500$&$0.493$&&&2&$0.028$&$0.723$&$1.00$&$0.000$&$0.000$\\
&3&$0.954$&$1.000$&$1.00$&$0.500$&$0.406$&&&3&$0.008$&$0.601$&$1.00$&$0.000$&$0.000$\\
&4&$0.194$&$0.985$&$1.00$&$0.018$&$0.001$&&&4&$0.963$&$1.000$&$1.00$&$0.500$&$0.489$\\
&5&$0.016$&$0.861$&$1.00$&$0.000$&$0.000$&&&5&$0.940$&$1.000$&$1.00$&$0.471$&$0.210$\\
&6&$0.973$&$1.000$&$1.00$&$0.500$&$0.740$&&&6&$0.960$&$1.000$&$1.00$&$0.499$&$0.334$\\
&7&$0.001$&$0.104$&$1.00$&$0.000$&$0.000$&&&7&$0.977$&$1.000$&$1.00$&$0.500$&$0.934$\\
&8&$0.947$&$1.000$&$1.00$&$0.462$&$0.191$&&&8&$0.957$&$1.000$&$1.00$&$0.496$&$0.301$\\
&9&$0.968$&$1.000$&$1.00$&$0.500$&$0.422$&&&9&$0.920$&$1.000$&$1.00$&$0.401$&$0.113$\\
\cline{1-7} \cline{9-15} 

\end{tabular}

}

\end{center}

\vspace{-1.2cm}

\begin{center}

\resizebox{\linewidth}{!}{

\begin{tabular}{|c|c|c|c|c|c|c|c|c|c|c|c|c|c|c|}

\cline{2-7} \cline{10-15} 

\multicolumn{1}{c}{} &\multicolumn{1}{|c|}{\multirow{3}{*}{Digit}} & \multicolumn{3}{c|}{$P{(H_0|{x})}$} &  \multicolumn{1}{c|}{\multirow{3}{*}{${\underline{P}}{(H_0|{x})}$}} & \multicolumn{1}{c|}{\multirow{3}{*}{${p_{\text{\textbf{obs}}}}$}} & \multicolumn{1}{c}{}&\multicolumn{1}{c|}{}& \multicolumn{1}{c|}{\multirow{3}{*}{Digit}} & \multicolumn{3}{c|}{$P{(H_0|{x})}$} &  \multicolumn{1}{c|}{\multirow{3}{*}{${\underline{P}}{(H_0|{x})}$}} & \multicolumn{1}{c|}{\multirow{3}{*}{${p_{\text{\textbf{obs}}}}$}}\\

\cline{3-5} \cline{11-13} 

\multicolumn{1}{c|}{}&&\multicolumn{1}{c|}{$a=1$}&\multicolumn{1}{c|}{$a = \theta_0$}&\multicolumn{1}{c|}{$a = 22 \, \theta_0$}& &&\multicolumn{1}{c}{}&  & &\multicolumn{1}{c|}{$a=1$}&\multicolumn{1}{c|}{$a = \theta_0$}&\multicolumn{1}{c|}{$a = 22 \, \theta_0$}& & \\

\multicolumn{1}{c|}{} & &\multicolumn{1}{c|}{$b=1$}&\multicolumn{1}{c|}{$b = 1-\theta_0$}&\multicolumn{1}{c|}{$b = 22-22 \, \theta_0$}& &&\multicolumn{1}{c}{}&  & &\multicolumn{1}{c|}{$b=1$}&\multicolumn{1}{c|}{$b = 1-\theta_0$}&\multicolumn{1}{c|}{$b = 22-22 \, \theta_0$}& &\\

\cline{1-7} \cline{9-15} 
\multirow{9}{*}{\rotatebox[origin=c]{90}{\textbf{Ireland BL1}}}&1&$0.894$&$0.993$&$1.00$&$0.460$&$0.186$ &&\multirow{9}{*}{\rotatebox[origin=c]{90}{\textbf{Luxembourg BL1}}}&1&$0.000$&$0.000$&$1.00$&$0.000$&$0.000$\\
&2&$0.957$&$0.999$&$1.00$&$0.500$&$0.599$&&&2&$0.046$&$0.745$&$1.00$&$0.018$&$0.001$\\
&3&$0.772$&$0.998$&$1.00$&$0.275$&$0.045$&&&3&$0.921$&$0.999$&$1.00$&$0.460$&$0.187$\\
&4&$0.015$&$0.817$&$1.00$&$0.000$&$0.000$&&&4&$0.966$&$1.000$&$1.00$&$0.500$&$0.663$\\
&5&$0.453$&$0.993$&$1.00$&$0.111$&$0.010$&&&5&$0.000$&$0.005$&$1.00$&$0.000$&$0.000$\\
&6&$0.055$&$0.962$&$1.00$&$0.000$&$0.000$&&&6&$0.966$&$1.000$&$1.00$&$0.500$&$0.494$\\
&7&$0.948$&$1.000$&$1.00$&$0.472$&$0.213$&&&7&$0.955$&$1.000$&$1.00$&$0.489$&$0.264$\\
&8&$0.872$&$0.999$&$1.00$&$0.330$&$0.067$&&&8&$0.894$&$1.000$&$1.00$&$0.364$&$0.086$\\
&9&$0.802$&$0.999$&$1.00$&$0.256$&$0.039$&&&9&$0.977$&$1.000$&$1.00$&$0.500$&$0.690$\\
\cline{1-7} \cline{9-15} 

\end{tabular}

}

\end{center}

\vspace{-1.2cm}

\begin{center}

\resizebox{\linewidth}{!}{
\begin{tabular}{|c|c|c|c|c|c|c|c|c|c|c|c|c|c|}

\cline{2-7} \cline{9-14} 

\multicolumn{1}{c}{} &\multicolumn{1}{|c|}{\multirow{3}{*}{Digit}} & \multicolumn{3}{c|}{$P{(H_0|{x})}$} &  \multicolumn{1}{c|}{\multirow{3}{*}{${\underline{P}}{(H_0|{x})}$}} & \multicolumn{1}{c|}{\multirow{3}{*}{${p_{\text{\textbf{obs}}}}$}} & & \multicolumn{1}{c|}{\multirow{3}{*}{Digit}} & \multicolumn{3}{c|}{$P{(H_0|{x})}$} &  \multicolumn{1}{c|}{\multirow{3}{*}{${\underline{P}}{(H_0|{x})}$}} & \multicolumn{1}{c|}{\multirow{3}{*}{${p_{\text{\textbf{obs}}}}$}}\\

\cline{1-1} \cline{3-5} \cline{10-12} 

\multirow{7}{*}{\rotatebox[origin=c]{90}{\textbf{Portugal BL1}}}&&\multicolumn{1}{c|}{$a=1$}&\multicolumn{1}{c|}{$a = \theta_0$}&\multicolumn{1}{c|}{$a = 22 \, \theta_0$}& & & & &\multicolumn{1}{c|}{$a=1$}&\multicolumn{1}{c|}{$a = \theta_0$}&\multicolumn{1}{c|}{$a = 22 \, \theta_0$}& & \\

& &\multicolumn{1}{c|}{$b=1$}&\multicolumn{1}{c|}{$b = 1-\theta_0$}&\multicolumn{1}{c|}{$b = 22-22 \, \theta_0$}& & & & &\multicolumn{1}{c|}{$b=1$}&\multicolumn{1}{c|}{$b = 1-\theta_0$}&\multicolumn{1}{c|}{$b = 22-22 \, \theta_0$}& &\\

\cline{2-7} \cline{9-14}
&1&$0.000$&$0.000$&$1.00$&$0.000$&$0.000$&&6&$0.974$&$1.000$&$1.00$&$0.500$&$0.848$\\
&2&$0.001$&$0.031$&$1.00$&$0.000$&$0.000$&&7&$0.881$&$0.999$&$1.00$&$0.349$&$0.077$\\
&3&$0.964$&$1.000$&$1.00$&$0.500$&$0.662$&&8&$0.972$&$1.000$&$1.00$&$0.500$&$0.591$\\
&4&$0.001$&$0.112$&$1.00$&$0.000$&$0.000$&&9&$0.979$&$1.000$&$1.00$&$0.500$&$0.959$\\
&5&$0.972$&$1.000$&$1.00$&$0.500$&$0.726$&&-&-&-&-&-&-\\
\cline{1-7} \cline{9-14}

\end{tabular}

}

\end{center}

\vspace{-.8em}

}

\block{Conclusion}{

The irreconcilability of classical and Bayesian measures of evidence in precise null hypothesis testing \citep{bergerSelke1987} arises in digit analysis. Even lower bounds on $P(H_0|\text{data})$, which are biased towards $H_1$, often provide much more evidence in favour of $H_0$  than what $p$-values seem to suggest. One incurring in the $p$-value fallacy \citep{goodman1999a} will typically overestimate the practical importance of an observed deviation from Benford's law. Classical tests of fixed dimension are of limited usefulness in digit analysis: they attain high levels of power 
in large samples and hence tiny deviations from Benford's law, without practical importance, are likely to be statistically significant. Conclusions drawn from classical tests, and agreement with Bayesian model selection, are sensible to an arbitrarily chosen evidence threshold. It can therefore be misleading to draw sharp conclusions based solely on statistical significance. Conclusions based solely on $p$-values, or obtained with the conventional $0.05$ threshold, are likely to underestimate the evidence in favour of Benford's law. 

}
    
\block{References}{
\vspace{-1em}

\printbibliography[heading=none]

}

\end{columns}

\vspace{1em}

\node [above right,
       outer sep=0pt,
       minimum width=.95\paperwidth,
       minimum height=2cm,
       align=center,
       font=\footnotesize] at ([shift={(0.5*\pgflinewidth,0.5*\pgflinewidth)}]bottomleft) {For data, code, replicability, and supplementary materials see \href{https://github.com/pedro-teles-fonseca/bayes-digit-analysis-warwick}{https://github.com/pedro-teles-fonseca/bayes-digit-analysis-warwick}};
       
\end{document}





